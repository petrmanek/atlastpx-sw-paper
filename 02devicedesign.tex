%%% Article: Software System for Data Acquisition and Analysis Operating the ATLAS-TPX Network
%%% Authors: Petr Manek, Jakub Begera
%%% Copyright (c) 2017 IEAP CTU


\section{\label{sec:device}Device Design}
Each ATLAS-TPX device consists of two Timepix~\cite{Llopart2007} readout chips with silicon sensor layers of thicknesses 300\,$\mu$m and 500\,$\mu$m facing each other~\cite{Bergmann_ATLASTPX_2016}. They are interlaced by a set of neutron converters. The Timepix ASIC (application specific integrated circuit) divides the sensor area into a square matrix of $256 \times 256$ contiguous pixels with a pixel dimension of 55\,$\mu$m. It allows a configuration of each pixel in either of the three modes of operation: 
~
\begin{itemize}
  \item In the spectroscopic Time-over-Threshold (ToT) mode the energy deposition in the sensor material is measured.
  \item In the Time-of-Arrival (ToA) mode the time from an interaction with respect to the end of the exposure is recorded (precision up to 25\,ns).
  \item In the counting mode, the number of interactions with energies above 5\,keV during the exposure time are counted.
\end{itemize}

Data are taken in so-called frames, representing the counter contents of all individual pixels after an adjustable exposure time (often also referred to as frame acquisition time). In each frame, interacting quanta of ionizing radiation can be seen as tracks on the pixel matrix, which have characteristic shapes, depending on the particle range in silicon, its deposited energy, angle of incidence, and particle type~\cite{Holy2008}.
